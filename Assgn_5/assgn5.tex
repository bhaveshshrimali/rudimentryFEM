\documentclass[11pt]{article}
\usepackage[margin=0.65in, paperwidth=8.5in, paperheight=11in]{geometry}
\usepackage{amsfonts}
\usepackage{pgfplotstable}
\usepackage{amsmath}
\usepackage{amsbsy}
\usepackage{authblk}
\usepackage{graphicx}
\usepackage{listings}
\usepackage{array}
\usepackage{titlesec}
\newcommand{\bs}[1]{\boldsymbol{#1}}
\newcommand{\del}[2]{\frac{\partial {#1}}{\partial {#2}}}
\usepackage{amssymb}
\usepackage{bm}
\usepackage{mathtools}
\usepackage{titlesec}
\titlespacing\section{10pt}{14pt plus 4pt minus 2pt}{10pt plus 2pt minus 2pt}
\titlespacing\subsection{0pt}{12pt plus 4pt minus 2pt}{8pt plus 2pt minus 2pt}
\titlespacing\subsubsection{0pt}{12pt plus 4pt minus 2pt}{6pt plus 2pt minus 2pt}
\usepackage{color} %red, green, blue, yellow, cyan, magenta, black, white
\usepackage{tikz}
\usetikzlibrary{shapes.geometric}
\usetikzlibrary{arrows.meta,arrows}
\tikzstyle{startstop}=[rectangle, rounded corners, minimum width=3cm, minimum height=1cm, text centered, draw-black, fill=red!30]
\tikzstyle{io}=[trapezium, trapezium left angle=70, trapezium right angle=110, minimum width=3cm, minimum height=1cm, text centered, draw-black, fill=red!30]
\tikzstyle{process} = [rectangle, minimum width=3cm, minimum height=1cm, text centered, draw=black, fill=orange!30]
\tikzstyle{decision} = [diamond, minimum width=3cm, minimum height=1cm, text centered, draw=black, fill=green!30]
\tikzstyle{arrow}=[thick, -> ,>=stealth]
\usepackage{tikz}
\usetikzlibrary{shapes.geometric}
\usetikzlibrary{arrows.meta,arrows}
\title{\bf Solution Procedures: CEE 570, Assignment 6}
\author{Bhavesh Shrimali, UIN: 652197102}
\begin{document}
\section{Newton-Raphson Method}
\setlength\parindent{0pt}
\begin{itemize}
\item Begin a general step (n). At this stage we have $||{\bf R_{step}}^i||$
\item Set the iteration counter to an initial value of zero (i=0)
\item Using the previously calculated residual, we ascertain $\Delta {bf d}^i$ using the expression: $${\bf K}({\bf \Delta d}^i)=\bf R^i$$. It is to be noted that for the first iteration of a particular step, the residual value at the end of the previous step is used to calculate the consistent tangent.
\item Update ${\bf d^i}$ as: $$\bf d^{i+1}=d^{i}+\Delta d^i$$
\item Calculation of Strains, in both the portions of the bar, naturally follows the computation of $\bf d^i$
\item A check is performed, to ascertain whether either of the portions are in the elastic regime or plastic regime, and accordingly determine updated $\bf K_a$ and/or $\bf K_b$ and $\bf K=K_a+K_b$
\item Calculate $||{\bf R_{step}}^i||$ and the check for the convergence criteria. If satisfied move to the next step. If not, the iterate further in the same step.\\ 
\end{itemize}
\section{Modified Newton Method}
The first line of difference between the Newton-Raphson method and the Modified-Newton Method is that the {\bf K} remains constant throughout the step for the former.\\The procedure for the latter is summarized as follows:
\begin{itemize}
\item Entering a particular step (n).  At this stage we have $||{\bf R_{step}}^i||$
\item Set the iteration counter to an initial value of zero (i=0)
\item Using the previously calculated residual, we ascertain $\Delta {\bf d}^i$ using the expression: $${\bf K}({\bf \Delta d}^i)=\bf R^i$$. It is to be noted that for the first iteration of a particular step, the residual value at the end of the previous step is used to calculate the consistent tangent.
\item This $\bf K$ value remains constant throughout the step.
\item This is followed by strain, and consequently force, calculations. It is ascertained first whether each of the portions of the bar are in elastic/plastic regions. 
\item Calculation of the Residual: $\bf R^{i+1}=F_{ext}-\Sigma\ {F_{int}}^{i+1}$ 
\item Check for the Residual: $\bf ||R^{i+1}|| \lessgtr \varepsilon||R^{0}||$
\item If the residual step is not satisfied  the update: ${\bf d^i}$ as: $$\bf d^{i+1}=d^{i}+\Delta d^i$$ else move to the next step.
\end{itemize}
\section{Comparison between the Newton-Raphson and Modified Newton}
The following section reiterates the comparison between Newton-Raphson and Modified Newton-Method, as observed through the Q-3 in the Assignment. 
\begin{itemize}
\item The problem begins in the Step 2. This is a nonlinear step and hence, unlike Step 1, which is linear, the value of residual obtained at the end of the Step 2, and further, is nonzero for Modified Newton with $\varepsilon=10^{-12}$. 
\item Whereas in case of the Newton-Raphson Method the solution converges exactly after 2 steps and the value of Residual obtained at the end of the second step is exactly equal to zero. The value of displacement obtained in this case is $d=1.9803922x10^{-2}$
\item This can be attributed to the fact that the force-displacement curve is bi-linear and hence the value of updated tangent is exactly equal to the tangent-2 in the force-displacement curve. 
\item Contrary to what is observed above, the Modified-Newton Method takes 68 iterations to converge to a value less than the tolerance desired, and still is not exactly equal to zero. 
\item This can be attributed to the fact that the value of initial tangent can never be equal to the tangent-2, effectively saying $\bf E \neq E_T$
\item It can be concluded that the Newton-Raphson Method, gives much faster convergence. For more complicated and strongly nonlinear problems, the Modified-Newton Method would eventually prove to be computationally efficient, because it would save time for updating the stiffness matrix.
\end{itemize} 
\end{document}
